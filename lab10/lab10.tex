\documentclass[a4paper, 12pt]{article}

\usepackage[utf8]{inputenc}
\usepackage[T1, T2A]{fontenc}
\usepackage[english, russian]{babel}
\usepackage[left = 2cm, right = 2cm, bottom = 2cm, top = 2cm]{geometry}
\usepackage[top=2cm, bottom=2cm, left=2cm, right=2cm]{geometry}

\usepackage{pgfplots}
\usetikzlibrary{pgfplots.polar}
\pgfplotsset{compat=1.13}
\pgfplotsset{grid = major, grid style = {dashed}}
\usepackage{threeparttablex} 
\usepackage{booktabs} 
\usepackage[tableposition=top]{caption}

\usepackage{subcaption}
\usepackage{amsmath}
\newcommand\tline[2]{$\underset{\text{#1}}{\text{\underline{\hspace{#2}}}}$}
\usepackage{tabu}
\usepackage{indentfirst}
% PGFPlots Table ========================================================
\usepackage{pgfplotstable}
\renewcommand{\arraystretch}{1.3}
% recommended:
\usepackage{booktabs}
\usepackage{colortbl}
% pgfplotstable settings
\pgfplotstableset{
    every head row/.style = {before row = \hline},
    after row = {[1mm] \hline},
    column type = {|c},
    every last column/.style={
        column type/.add={}{|},
    },   
}
\usepackage{threeparttable}
\usepackage{tabu}
\usepackage{threeparttablex} 
\usepackage{booktabs} 
\usepackage[tableposition=top]{caption}

% Paragraph indent
\usepackage{indentfirst}
\setlength{\parindent}{15mm}

%Change label separator
\usepackage{caption}
\captionsetup[figure]{labelformat=simple, labelsep = endash, name = Рисунок}
%\captionsetup[table]{labelformat=simple, labelsep = endash, justification = raggedright, singlelinecheck = off, width = 0.4\textwidth}
\captionsetup[table]{labelformat=simple, labelsep = endash, justification = raggedright, singlelinecheck = off}
\DeclareCaptionLabelFormat{gostfigure}{Рисунок #2}
\DeclareCaptionLabelFormat{gosttable}{Таблица #2}

\begin{document}
	\parindent=1.27cm
	\begin{titlepage}
	\centering
	{\fontsize{12pt}{5cm}\selectfont \bfseries Министерство образования и науки Российской Федерации} \\ \vspace{0.5cm}
	{\fontsize{7pt}{5cm}\selectfont ФЕДЕРАЛЬНОЕ ГОСУДАРСТВЕННОЕ АВТОНОМНОЕ ОБРАЗОВАТЕЛЬНОЕ УЧРЕЖДЕНИЕ ВЫСШЕГО ПРОФЕССИОНАЛЬНОГО ОБРАЗОВАНИЯ} \\ 
	\vspace{1cm}
	{\fontsize{12pt}{5cm}\selectfont \bfseries САНКТ-ПЕТЕРБУРГСКИЙ УНИВЕРСИТЕТ ИНФОРМАЦИОННЫХ ТЕХНОЛОГИЙ, МЕХАНИКИ И ОПТИКИ} \\ \vspace{1.5cm}
	
	{\fontsize{14pt}{5cm}\selectfont Кафедра \hspace{1cm} \underline{Систем Управления и Информатики}  \hspace{1cm} Группа \underline{Р3340}} \\ 
	\vspace{2cm}
	
	{\fontsize{20pt}{5cm}\selectfont \bfseries Лабораторная работа №10} \\
	{\fontsize{20pt}{5cm}\selectfont \bfseries “ИССЛЕДОВАНИЕ МАТЕМАТИЧЕСКОЙ МОДЕЛИ
		ЭЛЕКТРОМЕХАНИЧЕСКОГО ОБЪЕКТА УПРАВЛЕНИЯ”} \\
	{\fontsize{14pt}{5cm}\selectfont Вариант - 1} \\
	\vspace{1.5cm}
	
	\flushleft
	
	{Выполнил \hspace{2cm} \tline{(фамилия, и.о.)}{9cm} (подпись)} \\
	\vspace{2cm}
	
	{Проверил \hspace{2cm} \tline{(фамилия, и.о.)}{9cm} (подпись)} \\
	\vspace{5cm}
	
	"\underline{\hspace{0.7cm}}"\hspace{0.2cm}\underline{\hspace{2cm}}\hspace{0.2cm}20\underline{\hspace{0.7cm}}г. \hspace{2cm} Санкт-Петербург, \hspace{2cm} 20\underline{\hspace{0.7cm}}г. \\ \vspace{1cm}
	
	Работа выполнена с оценкой \hspace{1cm} \underline{\hspace{8cm}} \\ 
	\vspace{1cm}
	Дата защиты "\underline{\hspace{0.7cm}}"\hspace{0.2cm}\underline{\hspace{2cm}}\hspace{0.2cm}20\underline{\hspace{0.7cm}}г.
	
\end{titlepage}


\section*{\centering Задание}
\textbf{Цель работы} - изучение математических моделей и исследование характеристик электромеханического объекта управления, построенного на основе электродвигателя постоянного тока независимого возбуждения. \par
Необходимо по известной модели электромеханического объекта (ЭМО) построить схему и провести математическое моделирование при различных параметрах системы. Функциональная схема исследумого объекта представлена на рисунке 1.
\begin{figure} [h!]
    \centering
    \includegraphics[width = 0.7\textwidth]{images/EMO.png}
    \caption{Функциональная схема ЭМО}
\end{figure} \par
Усилительно-преобразовательное устройство (УПУ) описывается следующим уравнением:
\begin{equation}
    T_y\dot{U_y} + U_y = K_yU
\end{equation} \par
УПУ подключается к электродвигателю (ЭД) - двигетлю постоянного тока (ДПТ), к которому пеодключен исполнительный механизм (ИМ) через редуктор (Р) с целью снизить момент на роторе двигателя. Описанную систему можно описать следующими уравнениями.
\begin{align}
    T_\text{я}\dot{I} + I & = K_\text{д}(U_y + K_e\omega_Mi_p) &
    K_\text{м} - \frac{M_\text{СМ}}{i_p} & = J_\Sigma\dot{\omega}_Mi &
    J_\Sigma & = J_\text{д} + J_p + \frac{J_M}{i_p^2}
\end{align} \par
Изменяя параметры $M_\text{СМ}$, $i_p$, $J_M$, $T_\text{я}$ и $T_y$ необходимо получить графики переходных процессов и сравнить их.

В таблице 1 представлены исходные данные для моделирования ДПТ.
\begin{table}[h!]
	\tabulinesep = 2mm
	\centering
	\begin{threeparttable}
 		\caption{ Исходные данные.}\label{tab:perflogcross}
   \begin{tabular}{|c|c|c|c|c|c|c|c|c|c|}
            \hline
            $U_\text{Н}$ & $n_0$ & $I_\text{Н}$ & $M_\text{Н}$ & $R$ & $T_\text{Я}$ & $J_\text{Д}$ & $T_\text{у}$ & $i_\text{р}$ & $J_\text{М}$\\
            В & об/мин & А & Н$\cdot$м & Ом & мс & кг$\cdot\text{м}^2$ & мс & &  кг$\cdot\text{м}^2$ \\ \hline
            27 & 600 & 1.4 & 0.6 & 6.6 & 6 & $1.5\cdot10^{-3}$ & 4 & 15 & 0.05 \\
            \hline
        \end{tabular}

    \end{threeparttable}    
\end{table}
\newpage
\begin{center}
\section{Рассчет параметров моделирования}
\end{center}
\par По исходным данным можно рассчитать некоторые параметры моделирования.\par
\begin{align*}
    K_y & = \frac{U_\text{Н}}{U_m} = \frac{27}{10} = 2.7 & w_0 & = n_0\frac{\pi}{30} = 62.8 \\
    K_e & = \frac{U_\text{Н}}{w_0} = 0.43 & K_\text{Д} & = \frac{1}{R} = 0.15 \\
    K_\text{М} & = \frac{M_\text{Н}}{I_\text{Н}} =  0.43 & J_{\Sigma} & = 1.2J_\text{Д} + \frac{J_\text{М}}{i^2_p} = 2 \cdot 10^{-3}
\end{align*} \par
Коэффициенты передачи измерительных устройств можно найти предварительно промоделировав систему и выбрав максимальное время моделирования. В итоге получим следующие значения коэффициентов: 
\begin{align*}
    K_U & = \frac{\hat{U}_{ymax}}{U_\text{Н}} = \frac{10}{27} = 0.37  & K_I & = \frac{\hat{I}_{max}}{I_{max}} = \frac{10}{1.1} =  9.09\\
    K_\omega & = \frac{\hat{\omega}_{0}}{\omega_{max}} = \frac{10}{62.8} = 0.16 & K_\alpha & = \frac{\hat{\alpha}_{max}}{\alpha_{max}} = \frac{10}{0.94} = 21.3
\end{align*}

\newpage
\begin{center}
\section{Вывод моделей ВСВ}
\end{center}
\subsection{Модель ВСВ полной модели ЭМО}
Для начала запишем все уравнения, описывающие работу ЭМО. Их возьмем из теории.

\begin{equation}
    \begin{cases}
        k_\text{м}I - M_c = J_\Sigma \dot{\omega} \\
        T_\text{я}\dot{I} + I = k_\text{д}U_y - k_\text{д}k_e\omega \\
        T_y\dot{U_y} + U_y = k_yU
    \end{cases} \Leftrightarrow
    \begin{cases}
        \dot{\omega} = \frac{k_\text{м}}{J_\Sigma}I - \frac{1}{J_\Sigma}M_c \\
        \dot{I} = - \frac{k_\text{д}k_e}{T_\text{я}}\omega - \frac{1}{T_\text{я}}I + \frac{k_\text{д}}{T_\text{я}}U_y \\
        \dot{U_y} = -\frac{1}{T_y}U_y + \frac{k_y}{T_y}U
    \end{cases}
\end{equation} \par
Теперь, приняв за вектор состояния $X = \begin{bmatrix} \alpha & \omega & I & U_y \end{bmatrix}^T$ и $\dot{\alpha} = \omega$, получим следующую модель вход состояние выход (ВСВ).

\begin{align}
    \begin{bmatrix}
        \dot{\alpha} \\
        \dot{\omega} \\
        \dot{I} \\
        \dot{U_y} 
    \end{bmatrix} & = 
    \begin{bmatrix}
        0 & 1 & 0 & 0 \\
        0 & 0 & \frac{k_\text{м}}{J_\Sigma} & 0 \\
        0 & -\frac{k_\text{д}k_e}{T_\text{я}} & - \frac{1}{T_\text{я}} & \frac{k_\text{д}}{T_\text{я}} \\
        0 & 0 & 0 & -\frac{1}{T_y}
    \end{bmatrix}
    \begin{bmatrix}
        \alpha \\
        \omega \\
        I \\
        U_y 
    \end{bmatrix} + 
    \begin{bmatrix}
        0 & 0 \\
        0 & - \frac{1}{J_\Sigma} \\
        0 & 0 \\
        \frac{k_y}{T_y} & 0
    \end{bmatrix}
    \begin{bmatrix}
        U(t) \\
        M_c(t)
    \end{bmatrix} \\
    \alpha & = 
    \begin{bmatrix}
        1 & 0 & 0 & 0
    \end{bmatrix}
    \begin{bmatrix}
        \alpha \\
        \omega \\
        I \\
        U_y 
    \end{bmatrix}
\end{align}

\subsection{Модель ВСВ упрощенной модели ЭМО}
\par Приравнивая в выражениях (3) $T_\text{я}$ и $T_y$ к 0. Получим следующие выражения:

\begin{equation}
    \begin{cases}
    \dot{\alpha} = \omega \\
    \dot{\omega} = -\frac{k_\text{м}k_\text{д}k_e}{J_\Sigma}\omega + \frac{k_\text{м}k_\text{д}k_y}{J_\Sigma}U - \frac{1}{J_\Sigma}M_c
    \end{cases}
\end{equation}

И соответственно модель ВСВ: 
\begin{align}
    \begin{bmatrix}
        \dot{\alpha} \\
        \dot{\omega} \\
    \end{bmatrix} & = 
    \begin{bmatrix}
        0 & 1 \\
        0 & -\frac{k_\text{м}k_\text{д}k_e}{J_\Sigma} \\
    \end{bmatrix}
    \begin{bmatrix}
        \alpha \\
        \omega \\
    \end{bmatrix} + 
    \begin{bmatrix}
        0 & 0 \\
        \frac{k_\text{м}k_\text{д}k_y}{J_\Sigma} & -\frac{1}{J_\Sigma} \\
    \end{bmatrix}
    \begin{bmatrix}
        U(t) \\
        M_c(t)
    \end{bmatrix} \\
    \alpha & = 
    \begin{bmatrix}
        1 & 0 
    \end{bmatrix}
    \begin{bmatrix}
        \alpha \\
        \omega \\
    \end{bmatrix}
\end{align}

\newpage
\begin{center}
\section{Моделирование полной модели ЭМО}
\end{center}
\par На рисунке 2 представлна полная модель ДПТ.
\begin{figure}[h!]
    \centering
    \includegraphics[width = \textwidth]{images/FullModel/full-model.pdf}
    \caption{Полная модель ЭМО}
\end{figure}

После построения модели и определения параметров моделирования можно получить графики и подсчитать соответственно время переходного процесса $t_\text{п}$, установившиеся угловую скорость $\omega_y$ и ток $I_y$.

\begin{align*}
    t_\text{п} & = 0.35 & \omega_y & = 10 & I_y & = 0.0031 \\
\end{align*}

Ниже предсавлены графки переходных процессов двигателя при $T_y = 6\cdot10^{-3}$ c и $T_\text{Я} = 3\cdot10^{-3}$ c.

\begin{figure}[h!]
    \centering
    \begin{tikzpicture}
        \begin{axis} [
            width = 0.7\textwidth,
            height = 7cm,
            xlabel = {$t$, c},
            ylabel = {$\omega$, 1/c},
            grid = major,
            grid style = {dashed},
            xmin = 0, xmax = 0.6,
        ]
            \addplot[blue, mark = none, thick, smooth, solid] table [x = t, y = w] {data/FullModel/TransPlot.dat};
            \addplot[blue, mark = none, thick, smooth, dashed] table [x = t, y = I] {data/FullModel/TransPlot.dat};
            \addplot[blue, mark = none, thick, smooth, dotted] table [x = t, y = alpha] {data/FullModel/TransPlot.dat};
            \addplot[blue, mark = none, thick, smooth, dashdotted] table [x = t, y = U] {data/FullModel/TransPlot.dat};
            \legend{$\omega$, $I$, $\alpha_M$,$U$};
        \end{axis}
    \end{tikzpicture}
    \caption{Переходные процессы в ДПТ}
\end{figure}
\newpage
\begin{center}
\section{Исследование влияния момента сопротивленя $M_\text{СМ}$}
\end{center}
\par На рисунке 5 представлены переходные процессы ДПТ при различных значениях нагрузочного момента $M_\text{СМ}$.


\begin{figure}[h!]
    \begin{subfigure}{0.5\textwidth}
        \centering
        \begin{tikzpicture} 
            \begin{axis} [
                width = 0.9\textwidth,
                ylabel = {$\omega$, 1/c},
                xlabel = {$t$, c},
                legend pos=south east,
                grid = major,
                grid style = {dashed},
                xmin = 0, xmax = 0.6,
                extra y ticks={0, 10.04, 7.72, 5.4, 3.09},
                extra y tick style={grid style={black, thick, dashed}},
                ytick = {0, 2},
            ]
                \addplot[blue, mark = none, thick, smooth, solid] table [x = t, y = w1] {data/FullModel/Moment-w.dat};
                \addplot[blue, mark = none, thick, smooth, dashed] table [x = t, y = w2] {data/FullModel/Moment-w.dat};
                \addplot[blue, mark = none, thick, smooth, dotted] table [x = t, y = w3] {data/FullModel/Moment-w.dat};
                \addplot[blue, mark = none, thick, smooth, densely dashed] table [x = t, y = w4] {data/FullModel/Moment-w.dat};
                \legend{$M_{cm} = 0$, $M_{cm} = 3$, $M_{cm} = 6$, $M_{cm} = 9$};
            \end{axis}
        \end{tikzpicture}
    \end{subfigure}
    \begin{subfigure}{0.5\textwidth}
        \centering
        \begin{tikzpicture} 
            \begin{axis} [
                width = 0.9\textwidth,
                ylabel = {$I$, A},
                xlabel = {$t$, c},
                legend pos=north east,
                grid = major,
                grid style = {dashed},
                xmin = 0, xmax = 0.6,
                extra y ticks={0, 2.78, 5.56, 8.34, 10},
                extra y tick style={grid style={black, thick, dashed}},
                ytick = {4},
            ]
                \addplot[blue, mark = none, thick, smooth, solid] table [x = t, y = I1] {data/FullModel/Moment-I.dat};
                \addplot[blue, mark = none, thick, smooth, dashed] table [x = t, y = I2] {data/FullModel/Moment-I.dat};
                \addplot[blue, mark = none, thick, smooth, dotted] table [x = t, y = I3] {data/FullModel/Moment-I.dat};
                \addplot[blue, mark = none, thick, smooth, densely dashed] table [x = t, y = I4] {data/FullModel/Moment-I.dat};
                \legend{$M_{cm} = 0$, $M_{cm} = 3$, $M_{cm} = 6$, $M_{cm} = 9$};
            \end{axis}
        \end{tikzpicture}
    \end{subfigure}
    \vspace{0.5cm}

    \begin{subfigure}{0.5\textwidth}
        \centering
        \begin{tikzpicture} 
            \begin{axis} [
                width = 0.9\textwidth,
                ylabel = {$\alpha_M$, \text{градусы}},
                xlabel = {$t$, c},
                legend pos=north west,
                grid = major,
                grid style = {dashed},
                xmin = 0, xmax = 0.6,
            ]
                \addplot[blue, mark = none, thick, smooth, solid] table [x = t, y = alpha1] {data/FullModel/Moment-alpha.dat};
                \addplot[blue, mark = none, thick, smooth, dashed] table [x = t, y = alpha2] {data/FullModel/Moment-alpha.dat};
                \addplot[blue, mark = none, thick, smooth, dotted] table [x = t, y = alpha3] {data/FullModel/Moment-alpha.dat};
                \addplot[blue, mark = none, thick, smooth, densely dashed] table [x = t, y = alpha4] {data/FullModel/Moment-alpha.dat};
                \legend{$M_{cm} = 0$, $M_{cm} = 3$, $M_{cm} = 6$, $M_{cm} = 9$};
            \end{axis}
        \end{tikzpicture}
    \end{subfigure}
    \begin{subfigure}{0.5\textwidth}
        \centering
        \begin{tikzpicture} 
            \begin{axis} [
                width = 0.9\textwidth,
                ylabel = {$U$, B},
                xlabel = {$t$, c},
                legend pos=south east,
                grid = major,
                grid style = {dashed},
                xmin = 0, xmax = 0.3,
            ]

                \addplot[blue, mark = none, thick, smooth, solid] table [x = t, y = U1] {data/FullModel/Moment-U.dat};
            \end{axis}
        \end{tikzpicture}
    \end{subfigure}
    \caption{Графики прехеходных процессов при различных $M_\text{СМ}$}
\end{figure}

В ходе эксперимента, изменяя нагрузочный момент, мы получили различные значения времени переходного процесса и установившиеся значения тока и угловой скорости, которые представлены в таблице ниже.

\begin{table}[h!]
	\tabulinesep = 2mm
	\centering
	\begin{threeparttable}
        \caption{Данные о перехоных процессах}
        \pgfplotstabletypeset[
            columns/t_p_w/.style = {column name = {$t_\text{п}$}},
            columns/w/.style = {column name = {$\omega_y$}},
            columns/t_p_I/.style = {column name = {$t_\text{п}$}},
            columns/I/.style = {column name = {$I_y$}},
            columns/M/.style = {column name = {$M_\text{СМ}$}},
        ]{data/FullModel/Moment.dat}
 	\end{threeparttable}
\end{table}

\newpage
\begin{center}
	

\section{Исследование влеяния момента инерции нагрузки $J_\text{М}$}
\end{center}
\par На рисунке 5 представлены графики переходных процессов при различных значениях момента инерции нагрузки $J_\text{М}$.

\begin{figure}[h!]
    \begin{subfigure}{0.5\textwidth}
        \centering
        \begin{tikzpicture} 
            \begin{axis} [
                width = 0.9\textwidth,
                ylabel = {$\omega$, 1/c},
                xlabel = {$t$, c},
                legend pos=south east,
                grid = major,
                grid style = {dashed},
                xmin = 0, xmax = 0.6,
            ]
                \addplot[blue, mark = none, thick, smooth, solid] table [x = t, y = w1] {data/FullModel/InertiaMoment-w.dat};
                \addplot[blue, mark = none, thick, smooth, dashed] table [x = t, y = w2] {data/FullModel/InertiaMoment-w.dat};
                \addplot[blue, mark = none, thick, smooth, dotted] table [x = t, y = w3] {data/FullModel/InertiaMoment-w.dat};
                \addplot[blue, mark = none, thick, smooth, densely dashed] table [x = t, y = w4] {data/FullModel/InertiaMoment-w.dat};
                \legend{$J_{M} = 0.025$, $J_{M} = 0.05$, $J_{M} = 0.075$, $J_{M} = 0.01$};
            \end{axis}
        \end{tikzpicture}
    \end{subfigure}
    \begin{subfigure}{0.5\textwidth}
        \centering
        \begin{tikzpicture} 
            \begin{axis} [
                width = 0.9\textwidth,
                ylabel = {$I$, A},
                xlabel = {$t$, c},
                legend pos=north east,
                grid = major,
                grid style = {dashed},
                xmin = 0, xmax = 0.6,
            ]
                \addplot[blue, mark = none, thick, smooth, solid] table [x = t, y = I1] {data/FullModel/InertiaMoment-I.dat};
                \addplot[blue, mark = none, thick, smooth, dashed] table [x = t, y = I2] {data/FullModel/InertiaMoment-I.dat};
                \addplot[blue, mark = none, thick, smooth, dotted] table [x = t, y = I3] {data/FullModel/InertiaMoment-I.dat};
                \addplot[blue, mark = none, thick, smooth, densely dashed] table [x = t, y = I4] {data/FullModel/InertiaMoment-I.dat};
               
                \legend{$J_{M} = 0.025$, $J_{M} = 0.05$, $J_{M} = 0.075$, $J_{M} = 0.01$};
            \end{axis}
        \end{tikzpicture}
    \end{subfigure}
    \vspace{0.5cm}

    \begin{subfigure}{0.5\textwidth}
        \centering
        \begin{tikzpicture} 
            \begin{axis} [
                width = 0.9\textwidth,
                ylabel = {$\alpha_M$, \text{градусы}},
                xlabel = {$t$, c},
                legend pos=north west,
                grid = major,
                grid style = {dashed},
                xmin = 0, xmax = 0.6,
            ]
                \addplot[blue, mark = none, thick, smooth, solid] table [x = t, y = alpha1] {data/FullModel/InertiaMoment-alpha.dat};
                \addplot[blue, mark = none, thick, smooth, dashed] table [x = t, y = alpha2] {data/FullModel/InertiaMoment-alpha.dat};
                \addplot[blue, mark = none, thick, smooth, dotted] table [x = t, y = alpha3] {data/FullModel/InertiaMoment-alpha.dat};
                \addplot[blue, mark = none, thick, smooth, densely dashed] table [x = t, y = alpha4] {data/FullModel/InertiaMoment-alpha.dat};
                
               \legend{$J_{M} = 0.025$, $J_{M} = 0.05$, $J_{M} = 0.075$, $J_{M} = 0.01$};
            \end{axis}
        \end{tikzpicture}
    \end{subfigure}
    \begin{subfigure}{0.5\textwidth}
        \centering
        \begin{tikzpicture} 
            \begin{axis} [
                width = 0.9\textwidth,
                ylabel = {$U$, B},
                xlabel = {$t$, c},
                legend pos=south east,
                grid = major,
                grid style = {dashed},
                xmin = 0, xmax = 0.6,
            ]
                \addplot[blue, mark = none, thick, smooth, solid] table [x = t, y = U1] {data/FullModel/InertiaMoment-U.dat};
            \end{axis}
        \end{tikzpicture}
    \end{subfigure}
    \caption{Графики прехеходных процессов при различных $J_\text{М}$}
\end{figure}

В ходе эксперимента, изменяя момент инерции нагрузки, мы получили различные значения времени переходного процесса и установившиеся значения тока и угловой скорости, которые представлены в таблице ниже.

\begin{table}[h!]
	\tabulinesep = 2mm
	\centering
   \begin{threeparttable}
        \caption{Данные о перехоных процессах }
        \pgfplotstabletypeset[
            columns/t_p_w/.style = {column name = {$t_\text{п}$}},
            columns/w/.style = {column name = {$\omega_y$}},
            columns/t_p_I/.style = {column name = {$t_\text{п}$}},
            columns/I/.style = {column name = {$I_y$}},
            columns/J_m/.style = {column name = {$J_\text{М}$}},]{data/FullModel/InertiaMoment.dat}
    \end{threeparttable}
\end{table}

\newpage
\begin{center}
\section{Исследование влияния передаточного отношения $i_p$ редукотора}
\end{center}
\par На рисунке 6 представлены графики преходных процессов при различных значениях передаточного отношения и нулевом моменте нагрузки $M_\text{СМ} = 0$.

\begin{figure}[h!]
    \begin{subfigure}{0.5\textwidth}
        \centering
        \begin{tikzpicture} 
            \begin{axis} [
                width = 0.9\textwidth,
                ylabel = {$\omega$, 1/c},
                xlabel = {$t$, c},
                legend pos=south east,
                grid = major,
                grid style = {dashed},
                xmin = 0, xmax = 0.6,
                extra y ticks={5},
                extra tick style={grid=major},
            ]
                \addplot[blue, mark = none, thick, smooth, solid] table [x = t, y = w1] {data/FullModel/GearRatio-w.dat};
                \addplot[blue, mark = none, thick, smooth, dashed] table [x = t, y = w2] {data/FullModel/GearRatio-w.dat};
                \addplot[blue, mark = none, thick, smooth, dotted] table [x = t, y = w3] {data/FullModel/GearRatio-w.dat};
                \addplot[blue, mark = none, thick, smooth, densely dashed] table [x = t, y = w4] {data/FullModel/GearRatio-w.dat};
               
                \legend{$i_{p} = 15$, $i_{p} = 26.25$, $i_{p} = 37.5$, $i_{p} = 48.75$};
            \end{axis}
        \end{tikzpicture}
    \end{subfigure}
    \begin{subfigure}{0.5\textwidth}
        \centering
        \begin{tikzpicture} 
            \begin{axis} [
                width = 0.9\textwidth,
                ylabel = {$I$, A},
                xlabel = {$t$, c},
                legend pos=north east,
                grid = major,
                grid style = {dashed},
                xmin = 0, xmax = 0.6,
            ]
                \addplot[blue, mark = none, thick, smooth, solid] table [x = t, y = I1] {data/FullModel/GearRatio-I.dat};
                \addplot[blue, mark = none, thick, smooth, dashed] table [x = t, y = I2] {data/FullModel/GearRatio-I.dat};
                \addplot[blue, mark = none, thick, smooth, dotted] table [x = t, y = I3] {data/FullModel/GearRatio-I.dat};
                \addplot[blue, mark = none, thick, smooth, densely dashed] table [x = t, y = I4] {data/FullModel/GearRatio-I.dat};
                
                \legend{$i_{p} = 15$, $i_{p} = 26.25$, $i_{p} = 37.5$, $i_{p} = 48.75$};
            \end{axis}
        \end{tikzpicture}
    \end{subfigure}
    \vspace{0.5cm}

    \begin{subfigure}{0.5\textwidth}
        \centering
        \begin{tikzpicture} 
            \begin{axis} [
                width = 0.9\textwidth,
                ylabel = {$\alpha_M$, \text{градусы}},
                xlabel = {$t$, c},
                legend pos=north west,
                grid = major,
                grid style = {dashed},
                xmin = 0, xmax = 0.6,
            ]
                \addplot[blue, mark = none, thick, smooth, solid] table [x = t, y = alpha1] {data/FullModel/GearRatio-alpha.dat};
                \addplot[blue, mark = none, thick, smooth, dashed] table [x = t, y = alpha2] {data/FullModel/GearRatio-alpha.dat};
                \addplot[blue, mark = none, thick, smooth, dotted] table [x = t, y = alpha3] {data/FullModel/GearRatio-alpha.dat};
                \addplot[blue, mark = none, thick, smooth, densely dashed] table [x = t, y = alpha4] {data/FullModel/GearRatio-alpha.dat};
               
                \legend{$i_{p} = 15$, $i_{p} = 26.25$, $i_{p} = 37.5$, $i_{p} = 48.75$};
            \end{axis}
        \end{tikzpicture}
    \end{subfigure}
    \begin{subfigure}{0.5\textwidth}
        \centering
        \begin{tikzpicture} 
            \begin{axis} [
                width = 0.9\textwidth,
                ylabel = {$U$, B},
                xlabel = {$t$, c},
                legend pos=south east,
                grid = major,
                grid style = {dashed},
                xmin = 0, xmax = 0.6,
            ]
                \addplot[blue, mark = none, thick, smooth, solid] table [x = t, y = U1] {data/FullModel/GearRatio-U.dat};
            \end{axis}
        \end{tikzpicture}
    \end{subfigure}
    \caption{Графики прехеходных процессов при различных $i_p$ и $M_\text{СМ} = 0$}
\end{figure}

В ходе эксперимента, изменяя момент передаточное отношение редукторы, мы получили различные значения времени переходного процесса и установившиеся значения тока и угловой скорости, которые представлены в таблице ниже.

\begin{table}[h]
	\tabulinesep = 2mm
	\centering
    \begin{threeparttable}
        \caption{Данные о перехоных процессах }
        \pgfplotstabletypeset[
            columns/t_p_w/.style = {column name = {$t_\text{п}$}},
            columns/w/.style = {column name = {$\omega_y$}},
            columns/t_p_I/.style = {column name = {$t_\text{п}$}},
            columns/I/.style = {column name = {$I_y$}},
            columns/i_p/.style = {column name = {$i_{p}$}}, ]{data/FullModel/GearRatio.dat}
    \end{threeparttable}
\end{table}
\newpage
На рисунке 7 представлены графики преходных процессов при различных значениях передаточного отношения и не нулевом моменте нагрузки $M_\text{СМ} = M_\text{Н}i_p/2$, при $i_p = 15$.

\begin{figure}[h!]
	\begin{subfigure}{0.5\textwidth}
		\centering
		\begin{tikzpicture} 
		\begin{axis} [
		width = 0.9\textwidth,
		ylabel = {$\omega$, 1/c},
		xlabel = {$t$, c},
		legend pos=south east,
		grid = major,
		grid style = {dashed},
		xmin = 0, xmax = 0.6,
		extra y ticks={6.57, 8.07, 8.98},
		extra y tick style={grid style={black, thick, dashed}},
		ytick = {0, 5, 10},
		]
		\addplot[blue, mark = none, thick, smooth, solid] table [x = t, y = w1] {data/FullModel/GearRatioWithMoment-w.dat};
		\addplot[blue, mark = none, thick, smooth, dashed] table [x = t, y = w2] {data/FullModel/GearRatioWithMoment-w.dat};
		\addplot[blue, mark = none, thick, smooth, dotted] table [x = t, y = w3] {data/FullModel/GearRatioWithMoment-w.dat};
		\addplot[blue, mark = none, thick, smooth, densely dashed] table [x = t, y = w4] {data/FullModel/GearRatioWithMoment-w.dat};
		 \legend{$i_{p} = 15$, $i_{p} = 26.25$, $i_{p} = 37.5$, $i_{p} = 48.75$};
		\end{axis}
		\end{tikzpicture}
    \end{subfigure}
    \begin{subfigure}{0.5\textwidth}
        \centering
        \begin{tikzpicture} 
            \begin{axis} [
                width = 0.9\textwidth,
                ylabel = {$I$, A},
                xlabel = {$t$, c},
                legend pos=north east,
                grid = major,
                grid style = {dashed},
                xmin = 0, xmax = 0.6,
                extra y ticks={4.17, 2.37, 1.28},
                extra y tick style={grid style={black, thick, dashed}},
                ytick = {0, 6, 10},
            ]
                \addplot[blue, mark = none, thick, smooth, solid] table [x = t, y = I1] {data/FullModel/GearRatioWithMoment-I.dat};
                \addplot[blue, mark = none, thick, smooth, dashed] table [x = t, y = I2] {data/FullModel/GearRatioWithMoment-I.dat};
                \addplot[blue, mark = none, thick, smooth, dotted] table [x = t, y = I3] {data/FullModel/GearRatioWithMoment-I.dat};
                \addplot[blue, mark = none, thick, smooth, densely dashed] table [x = t, y = I4] {data/FullModel/GearRatioWithMoment-I.dat};
               
                 \legend{$i_{p} = 15$, $i_{p} = 26.25$, $i_{p} = 37.5$, $i_{p} = 48.75$};
            \end{axis}
        \end{tikzpicture}
    \end{subfigure}
    \vspace{0.5cm}

    \begin{subfigure}{0.5\textwidth}
        \centering
        \begin{tikzpicture} 
            \begin{axis} [
                width = 0.9\textwidth,
                ylabel = {$\alpha_M$, \text{градусы}},
                xlabel = {$t$, c},
                legend pos=north west,
                grid = major,
                grid style = {dashed},
                xmin = 0, xmax = 0.6,
            ]
                \addplot[blue, mark = none, thick, smooth, solid] table [x = t, y = alpha1] {data/FullModel/GearRatioWithMoment-alpha.dat};
                \addplot[blue, mark = none, thick, smooth, dashed] table [x = t, y = alpha2] {data/FullModel/GearRatioWithMoment-alpha.dat};
                \addplot[blue, mark = none, thick, smooth, dotted] table [x = t, y = alpha3] {data/FullModel/GearRatioWithMoment-alpha.dat};
                \addplot[blue, mark = none, thick, smooth, densely dashed] table [x = t, y = alpha4] {data/FullModel/GearRatioWithMoment-alpha.dat};
               
                \legend{$i_{p} = 15$, $i_{p} = 26.25$, $i_{p} = 37.5$, $i_{p} = 48.75$};
            \end{axis}
        \end{tikzpicture}
    \end{subfigure}
    \begin{subfigure}{0.5\textwidth}
        \centering
        \begin{tikzpicture} 
            \begin{axis} [
                width = 0.9\textwidth,
                ylabel = {$U$, B},
                xlabel = {$t$, c},
                legend pos=south east,
                grid = major,
                grid style = {dashed},
                xmin = 0, xmax = 0.6,
            ]
                \addplot[blue, mark = none, thick, smooth, solid] table [x = t, y = U1] {data/FullModel/GearRatioWithMoment-U.dat};
            \end{axis}
        \end{tikzpicture}
    \end{subfigure}
    \caption{Графики прехеходных процессов при различных $i_p$ и $M_\text{СМ} = M_\text{Н}i_p/2$}
\end{figure}

В ходе эксперимента, изменяя момент передаточное отношение редукторы, мы получили различные значения времени переходного процесса и установившиеся значения тока и угловой скорости, которые представлены в таблице ниже.

\begin{table}[h!]
   \tabulinesep = 2mm
   \centering
   \begin{threeparttable}
        \caption{Данные о перехоных процессах }
        \pgfplotstabletypeset[
            columns/t_p_w/.style = {column name = {$t_\text{п}$}},
            columns/w/.style = {column name = {$\omega_y$}},
            columns/t_p_I/.style = {column name = {$t_\text{п}$}},
            columns/I/.style = {column name = {$I_y$}},
            columns/i_p/.style = {column name = {$i_{p}$}}, ]{data/FullModel/GearRatioWithMoment.dat}
            \end{threeparttable}          
\end{table}

\newpage
\begin{center}
\section{Сравнение плоной и упрощенной модели ЭМО}
\end{center}
\par Моделируемая система изображена на рисунке ниже.

\begin{figure}[h!]
    \centering
    \includegraphics[width = 0.7\textwidth]{images/EasyModel/easy-model.pdf}
    \caption{Упрощенная модель ЭМО}
\end{figure}
\begin{align}
K = \frac{K_y}{k_e i_p}=\frac{(2.7)}{0.43.15}=0.42
\end{align}
\begin{align}
K_f = \frac{R}{k_M k_e i_p^2}=0.16
\end{align}
\begin{align}
T_M = \frac{RJ}{k_M k_e}=0.071
\end{align}
\subsection{Сравнение моделей при при $T_\text{я} = 6\cdot10^{-3}$ и $T_\text{у} = 4\cdot10^{-3}$}
Ниже указаны характеристики переходного процесса упрощенной модели ЭМО. А также представлен график, в котором сравниваются полная и упрощенная модель.
\begin{align*}
    t_\text{п} & = 0.35 & \omega_y & = 10 \\
\end{align*}

\begin{figure}[h!]
    \centering
    \begin{tikzpicture}
        \begin{axis} [
            width = 0.85\textwidth,
            height = 7cm,
            xlabel = {$t$, c},
            ylabel = {$\omega$, 1/c},
            grid = major,
            grid style = {dashed},
            legend pos = south east,
            legend style = {draw = none},
            xmin = 0, xmax = 0.6,
        ]
            \addplot[blue, mark = none, thick, smooth, solid] table [x = t, y = w] {data/FullModel/TransPlot.dat};
            \addplot[blue, mark = none, thick, smooth, dashed] table [x = t, y = w] {data/EasyModel/TransPlot.dat};
            \legend{Полная модель, Упрощенная модель модель};
        \end{axis}
    \end{tikzpicture}
    \caption{Сравенение переходных процессов угловой скорости $\omega$ упрощенной и полной модели ЭМО.}
\end{figure}

Отклонение упрощенной моедли от полной состалвяет:
\begin{equation}
    \Delta_{\omega1} = 0.05
\end{equation}

\newpage
\subsection{Сравнение моделей при $T_\text{я} = 6\cdot10^{-4}$ и $T_\text{у} = 4\cdot10^{-4}$}
Ниже представлен график, в котором сравниваются полная и упрощенная модель.

\begin{figure}[h!]
    \centering
    \begin{tikzpicture}
        \begin{axis} [
            width = 0.85\textwidth,
            height = 7cm,
            xlabel = {$t$, c},
            ylabel = {$\omega$, 1/c},
            grid = major,
            grid style = {dashed},
            legend pos = south east,
            legend style = {draw = none},
            xmin = 0, xmax = 0.6,
        ]
            \addplot[blue, mark = none, thick, smooth, solid] table [x = t, y = w] {data/FullModel/TransPlotSwitchedTime.dat};
            \addplot[blue, mark = none, thick, smooth, dashed] table [x = t, y = w] {data/EasyModel/TransPlotSwitchedTime.dat};
            \legend{Полная модель, Упрощенная модель};
        \end{axis}
    \end{tikzpicture}
    \caption{Сравенение переходных процессов угловой скорости $\omega$ упрощенной и полной модели ЭМО.}
\end{figure}

Отклонение упрощенной моедли от полной состалвяет:
\begin{equation}
    \Delta_{\omega1} = 0.001
\end{equation}

\newpage
\section*{\centering Выводы}
Мы были исследованы математические модели электромеханического объекта. В ходе работы были построены графики переходных процессов, исследованы влияния параметров объекта на вид этих процессов. При увеличении момента нагрузки $M_\text{СМ}$: уменьшается установившаяся угловая скорость двигателя и время переходного процесса, при этом увеличивается установившийся ток. При увеличении момента инерции нагрйзки: увеличивается время переходного процесса и максимальный ток. \par
При увеличении передаточного числа редуктора, уменьшается влияние момента инерции нагрузки и соответственно уменьшается время переходного процесса. Также уменьшается угловая скорость на выходе редуктора. \par
При увеличении передаточного числа редуктора увеличивается установившаяся угловая скорость (уменьшается ошибка) двигателя и уменьшается на выходе редуктора. Также уменьшается установившийся ток. \par
При сравнении графиков полной и упрощенной модели ЭМО,при уменьшении $T_\text{я}$ и $T_\text{у}$ уменьшается ошибка и график перехоная характеристика полной модели стремится к упрощенной. \par
\end{document}