\documentclass[a4paper, 12pt]{article}
\usepackage[utf8]{inputenc}
\usepackage[english,russian]{babel}
\usepackage[T1, T2A]{fontenc}
\usepackage{graphicx}

\usepackage{pgfplots}
\usetikzlibrary{pgfplots.polar}
\pgfplotsset{compat=1.13, grid=major}
\usepackage[left = 2cm, right = 2cm, bottom = 2cm, top = 2cm]{geometry}
\usepackage[top=2cm, left=2cm, right=2cm, left=2cm]{geometry}
\usepackage{amsmath}

\usepackage{tabu}
\usepackage{threeparttablex} 
\usepackage{booktabs} 
\usepackage[tableposition=top]{caption}

\usepackage{subcaption}
\DeclareCaptionLabelFormat{gostfigure}{Рисунок #2}
\DeclareCaptionLabelFormat{gosttable}{Таблица #2}
\DeclareCaptionLabelSeparator{gost}{~---~}
\captionsetup{labelsep=gost}
\captionsetup[figure]{labelformat=gostfigure}
\captionsetup[table]{labelformat=gosttable}
\renewcommand{\thesubfigure}{\asbuk{subfigure}}
\captionsetup[table]{labelformat=simple, labelsep = endash, justification = raggedright, singlelinecheck = off}
\usepackage{indentfirst}
\graphicspath{{image/}}
\newcommand\tline[2]{$\underset{\text{#1}}{\text{\underline{\hspace{#2}}}}$}

% PGFPlots Table ========================================================
\usepackage{pgfplotstable}
\renewcommand{\arraystretch}{1.5}
% pgfplotstable settings
\pgfplotstableset{
    columns/w/.style = {column name = {\boldmath$\omega$}, column type = |c},
    columns/lg_w/.style = {column name = {\boldmath$\lg{\omega}$}, column type = |c},
    columns/A/.style = {column name = {\boldmath$A(\omega)$}, column type = |c},
    columns/L/.style = {column name = {\boldmath$20\lg{A(\omega)}$}, column type = |c},
    columns/psi/.style = {column name = {\boldmath$\psi$}, column type = |c|},
    every head row/.style = {before row = \hline},
    after row = {[1mm] \hline},
}

\begin{document}
	\parindent=1.27cm
	\begin{titlepage}
		\centering
		{\fontsize{12pt}{5cm}\selectfont \bfseries Министерство образования и науки Российской Федерации} \\ \vspace{0.5cm}
		{\fontsize{7pt}{5cm}\selectfont ФЕДЕРАЛЬНОЕ ГОСУДАРСТВЕННОЕ АВТОНОМНОЕ ОБРАЗОВАТЕЛЬНОЕ УЧРЕЖДЕНИЕ ВЫСШЕГО ПРОФЕССИОНАЛЬНОГО ОБРАЗОВАНИЯ} \\ 
		\vspace{1cm}
		{\fontsize{12pt}{5cm}\selectfont \bfseries САНКТ-ПЕТЕРБУРГСКИЙ УНИВЕРСИТЕТ ИНФОРМАЦИОННЫХ ТЕХНОЛОГИЙ, МЕХАНИКИ И ОПТИКИ} \\ \vspace{1.5cm}

		{\fontsize{14pt}{5cm}\selectfont Кафедра \hspace{1cm} \underline{Систем Управления и Информатики}  \hspace{1cm} Группа \underline{Р3340}} \\ 
		\vspace{2cm}

		{\fontsize{20pt}{5cm}\selectfont \bfseries Лабораторная работа №9} \\
		{\fontsize{20pt}{5cm}\selectfont \bfseries “Экспериментальное построение частотных характеристик типовых динамических звеньев”} \\
		{\fontsize{14pt}{5cm}\selectfont Вариант - 1} \\
		\vspace{1.5cm}

		\flushleft

		{Выполнил \hspace{2cm} \tline{(фамилия, и.о.)}{9cm} (подпись)} \\
		\vspace{2cm}

		{Проверил \hspace{2cm} \tline{(фамилия, и.о.)}{9cm} (подпись)} \\
		\vspace{5cm}

		"\underline{\hspace{0.7cm}}"\hspace{0.2cm}\underline{\hspace{2cm}}\hspace{0.2cm}20\underline{\hspace{0.7cm}}г. \hspace{2cm} Санкт-Петербург, \hspace{2cm} 20\underline{\hspace{0.7cm}}г. \\ \vspace{1cm}

		Работа выполнена с оценкой \hspace{1cm} \underline{\hspace{8cm}} \\ 
		\vspace{1cm}
		Дата защиты "\underline{\hspace{0.7cm}}"\hspace{0.2cm}\underline{\hspace{2cm}}\hspace{0.2cm}20\underline{\hspace{0.7cm}}г.

\end{titlepage}

\begin{center}
\section*{Задание}
\end{center}

\subsection*{Цель работы} 
\par
Изучение частотных характеристик типовых динамических звеньев и способов их построения; построение частотных характеристик, расчёт передаточных функций для заданных типовых звеньев.
\par
В работе предстоит построить АЧХ, ФЧХ, АФЧХ и ЛАФЧХ исследуемых звеньев, а также асимптотические ЛАЧХ, построенные графо-аналитическим методом. На вход исследуемого звена подаётся синусоидальный сигнал постоянной амплитуды. Надо измерить амплитуду выходного сигнала и сдвиг фаз между входным и выходным сигналами при различных частотах - таким образом будут получены данные для построения частотных характеристик. 

\begin{table}[h!]
    %\tabulinesep = 2mm
    \centering
    \begin{threeparttable}
    	\caption{Исходные элементарные звенья}
    	\begin{tabular} {|l|c|}
        \hline
        	Тип звена & Передаточная функция \\ [0.5cm]  \hline
        	Апериодическое 1-го порядка & $\displaystyle\frac{k}{Ts + 1}$ \\ [0.5cm]  \hline
        	Дифференцирующее с замедлением & $\displaystyle\frac{ks}{Ts+1}$ \\ [0.5cm]  \hline
        	Интегрирующее с замедлением & $\displaystyle\frac{k}{s(Ts + 1)}$ \\ [0.5cm] \hline
    	\end{tabular}
    \end{threeparttable} 
\end{table}

\begin{table}[h!]
    \tabulinesep = 2mm
    \centering
    \begin{threeparttable}
    	\caption{Параметры}\label{tab:perflogcross}
    	\begin{tabular}{|c|c|c|}
    		\hline
        	k & T & $\xi$ \\ \hline
        	5 & 0.1 & 0.1 \\
        	\hline
    	\end{tabular}
    \end{threeparttable} 
\end{table}

\newpage
\begin{center}
	\section{Исследование апериодического 1-го порядка}
\end{center}

\par 
Передаточная функция исследуемого звена:
\begin{align}
	W(s)=\frac{k}{Ts+1}
\end{align}
\par 
Найдём выражения для АЧХ и ФЧХ:
\begin{align}
W(j\omega) = \frac{k(1-T\omega j)}{T^2\omega^2 + 1}
\end{align}

\begin{align}
	A(\omega) = \frac{k}{\sqrt{T^2\omega^2 + 1}}
\end{align}
	
\begin{align}
	\psi(\omega) =-\arctg{T\omega}
\end{align}

\par 
Экспериментальные данные, полученные по результатам моделирования, представлены в таблице 3.
\begin{table}[h!]
    \centering
    \begin{threeparttable}
        \caption{Полученные данные} \label{tab:perflogcross}
        \pgfplotstabletypeset[]{data/data1.txt}
    \end{threeparttable}
\end{table}

\newpage
\par 
На рисунке 1 представлены частотные характеристики интегрирующего звена с замедлением.

\begin{figure}[h!]
    \begin{subfigure}{0.5\textwidth}
        \centering
        \begin{tikzpicture}
            \begin{semilogxaxis} [
                    width = 0.9\textwidth,
                    xlabel = {$\omega$, 1/c},
                    ylabel = {$L_m$, дБ},
                    xmin = 10e-1,
                    xmax = 10e+2,
                   	ytick = {13.98, 10, 5, 0, -6.02},
            		extra x ticks = {0.2},
            		xtick= {10e-3, 10e-1, 10e+0, 10e+1}
                ]
                \addplot table [x={w}, y={L}] {data/data1.txt};
                \draw (1, 13.98) -- (10, 13.98) -- (100, -6.02);
               
            \end{semilogxaxis}
        \end{tikzpicture}
        \caption{График ЛАЧХ и асимптотическая ЛАЧХ}
    \end{subfigure}
    \begin{subfigure}{0.5\textwidth}
        \centering
        \begin{tikzpicture}
            \begin{semilogxaxis} [
                    width = 0.9\textwidth,
                    xlabel = {$\omega$, 1/c},
                    ylabel = {$\psi$, градусы},
                    ytick = {-5.71, -20, -40, -60, -84.29},
                ]
                \addplot table [x={w}, y={psi}] {data/data1.txt};
            \end{semilogxaxis}
        \end{tikzpicture}
        \caption{График ЛФЧХ}
    \end{subfigure}
    
    \vspace{0.5cm}

    \begin{subfigure}{0.5\textwidth}
        \centering
        \begin{tikzpicture}
            \begin{polaraxis} [
                    width = 0.9\textwidth,
                    xlabel = {$A(\omega)$},
                    ylabel = {$\psi$, градусы},
                ]
                \addplot table [x={psi}, y={A}] {data/data1.txt};
            \end{polaraxis}
        \end{tikzpicture}
        \caption{График АФЧХ}
    \end{subfigure}
    \begin{subfigure}{0.5\textwidth}
        \centering
        \begin{tikzpicture}
            \begin{polaraxis} [
                    width = 0.9\textwidth,
                    xlabel = {$L_m$, дБ},
                    ylabel = {$\psi$, градусы},
                ]
                \addplot table [x={psi}, y={L}] {data/data1.txt};
            \end{polaraxis}
        \end{tikzpicture}
        \caption{График ЛАФЧХ}
    \end{subfigure}
    \caption{Частотные характеристики интегрирующего звена с запаздыванием}
\end{figure}

\newpage
\begin{center}
	\section{Исследование дифференцирующего звена с замедлением}
\end{center}

\par 
Передаточная функция исследуемого звена:
\begin{align}
	W(s)=\frac{ks}{Ts+1}
\end{align}
\par 
Найдём выражения для АЧХ и ФЧХ:
\begin{align}
W(j\omega) = \frac{k\omega j}{1+T \omega j}
\end{align}

\begin{align}
	A(\omega) = \frac{k\omega}{\sqrt{T^2\omega^2 + 1}}
\end{align}

\begin{align}
\psi(\omega) = \arctg{\frac{1}{T\omega}}
\end{align}

\par 
Экспериментальные данные, полученные по результатам моделирования, представлены в таблице 4.

\newpage
\begin{table}[h!]
    \centering
    \begin{threeparttable}
        \caption{Полученные данные} \label{tab:perflogcross}
        \pgfplotstabletypeset[]{data/data2.txt}
    \end{threeparttable}
\end{table}

\newpage
\par 
На рисунке 2 представлены частотные характеристики дифференцирующего звена с замедлением

\begin{figure}[h!]
    \begin{subfigure}{0.5\textwidth}
        \centering
        \begin{tikzpicture}
            \begin{semilogxaxis} [
                    width = 0.9\textwidth,
                    xlabel = {$\omega$, 1/c},
                    ylabel = {$L_m$, дБ},
                    extra y ticks = { 13.93, 32},
                    extra x ticks = {0.2},
                    xtick = {10e-1, 10e+0, 10e+1}
                ]
                \addplot table [x={w}, y={L}] {data/data2.txt};
                \draw (1, 13.93) -- (10, 30.97) -- (100, 30.97);
            \end{semilogxaxis}
        \end{tikzpicture}
        \caption{График ЛАЧХ и асимптотическая ЛАЧХ}
    \end{subfigure}
    \begin{subfigure}{0.5\textwidth}
        \centering
        \begin{tikzpicture}
            \begin{semilogxaxis} [
                    width = 0.9\textwidth,
                    xlabel = {$\omega$, 1/c},
                    ylabel = {$\psi$, градусы},
                    	ytick = {5.71, 20, 40, 60, 84.29},
                ]
                \addplot table [x={w}, y={psi}] {data/data2.txt};
            \end{semilogxaxis}
        \end{tikzpicture}
        \caption{График ЛФЧХ}
    \end{subfigure}
    
    \vspace{0.5cm}

    \begin{subfigure}{0.5\textwidth}
        \centering
        \begin{tikzpicture}
            \begin{polaraxis} [
                    width = 0.9\textwidth,
                    xlabel = {$A(\omega)$},
                    ylabel = {$\psi$, градусы},
                ]
                \addplot table [x={psi}, y={A}] {data/data2.txt};
            \end{polaraxis}
        \end{tikzpicture}
        \caption{График АФЧХ}
    \end{subfigure}
    \begin{subfigure}{0.5\textwidth}
        \centering
        \begin{tikzpicture}
            \begin{polaraxis} [
                    width = 0.9\textwidth,
                    xlabel = {$L_m$, дБ},
                    ylabel = {$\psi$, градусы},
                ]
                \addplot table [x={psi}, y={L}] {data/data2.txt};
            \end{polaraxis}
        \end{tikzpicture}
        \caption{График ЛАФЧХ}
    \end{subfigure}
    \caption{Частотные характеристикидифференцирующего звена с замедлением}
\end{figure}

\newpage
\begin{center}
	\section{Исследование интегрирующего с замедлением }
\end{center}

\par 
Передаточная функция исследуемого звена:
\begin{align}
	W(s)=\frac{k}{Ts^2 + s}
\end{align}
\par 
Найдём выражения для АЧХ и ФЧХ:
\begin{align}
	W(j\omega) & = \frac{k}{j\omega(Tj\omega+1)}
\end{align}

\begin{align}
	A(\omega) & = \frac{k}{\omega(T^2\omega^2+1 )}
\end{align}

\begin{align}
\psi(\omega) & = -\pi/2-arctg(\omega T)
\end{align}

\par 
Экспериментальные данные, полученные по результатам моделирования, представлены в таблице 5.
\newpage
\begin{table}[h!]
    \centering
    \begin{threeparttable}
        \caption{Полученные данные} \label{tab:perflogcross}
        \pgfplotstabletypeset[]{data/data3.txt}
    \end{threeparttable}
\end{table}

\newpage
\par 
На рисунке 3 представлены частотные характеристики интегрирующего с замедлением.

\begin{figure}[h!]
    \begin{subfigure}{0.5\textwidth}
        \centering
        \begin{tikzpicture}
            \begin{semilogxaxis} [
                    width = 0.9\textwidth,
                    xlabel = {$\omega$, 1/c},
                    ylabel = {$L_m$, дБ},
                    ytick = { 13.94, 0, -10, -20, -30, -46.06 },
                    extra x ticks = {0.2},
                    xtick = {10e-3, 10e-1, 10e+0, 10e+1}
                ]
                \addplot table [x={w}, y={L}] {data/data3.txt};
                \draw (1, 13.94) -- (10, -6.06) -- (100, -46.06);
            \end{semilogxaxis}
        \end{tikzpicture}
        \caption{График ЛАЧХ и асимптотическая ЛАЧХ}
    \end{subfigure}
    \begin{subfigure}{0.5\textwidth}
        \centering
        \begin{tikzpicture}
            \begin{semilogxaxis} [
                    width = 0.9\textwidth,
                    xlabel = {$\omega$, 1/c},
                    ylabel = {$\psi$, градусы},
                    ytick = { 95.71, 120, 140, 160, 174.29 }
                ]
                \addplot table [x={w}, y={psi}] {data/data3.txt};
            \end{semilogxaxis}
        \end{tikzpicture}
        \caption{График ЛФЧХ}
    \end{subfigure}
    
    \vspace{0.5cm}

    \begin{subfigure}{0.5\textwidth}
        \centering
        \begin{tikzpicture}
            \begin{polaraxis} [
                    width = 0.9\textwidth,
                    xlabel = {$A(\omega)$},
                    ylabel = {$\psi$, градусы},
                ]
                \addplot table [x={psi}, y={A}] {data/data3.txt};
            \end{polaraxis}
        \end{tikzpicture}
        \caption{График АФЧХ}
    \end{subfigure}
    \begin{subfigure}{0.5\textwidth}
        \centering
        \begin{tikzpicture}
            \begin{polaraxis} [
                    width = 0.9\textwidth,
                    xlabel = {$L_m$, дБ},
                    ylabel = {$\psi$, градусы},
                ]
                \addplot table [x={psi}, y={L}] {data/data3.txt};
            \end{polaraxis}
        \end{tikzpicture}
        \caption{График ЛАФЧХ}
    \end{subfigure}
    \caption{Частотные характеристики интегрирующего с замедлением}
\end{figure}

\newpage
\begin{center}
	\section*{Вывод}
\end{center}
\par 
В лабораторной работе были исследованы следующие элементарные звенья: апериодическое 1-го порядка, дифференцирующее с замедлением и интегрирующее с замедлением. Были найдены частотные характеристики, а также построены графо-аналитическим методом асимптотические ЛАЧХ, к которым сходятся полученные экспериментально графики.
\end{document}
